%!TEX root = ../thesis.tex
%*******************************************************************************
%*********************************** First Chapter *****************************
%*******************************************************************************

\chapter{Introduction}  %Title of the First Chapter

\ifpdf
    \graphicspath{{Chapter1/Figs/Raster/}{Chapter1/Figs/PDF/}{Chapter1/Figs/}}
\else
    \graphicspath{{Chapter1/Figs/Vector/}{Chapter1/Figs/}}
\fi

\section{Breaking the diffraction limit reveals structure below 200\,nm}
Our understanding of the biology of life is limited to what we can observe. 
The human visual system is limited to a spatial resolution on the order of \textasciitilde\SI{100}{\micro\metre}~\cite{devalois1990spatial}.
The invention of high power microscopes, generally attributed to van Leeuwenhoek in 1660~\cite{van1800select}, and pioneering experiments by Robert Hooke~\cite{hooke1667micrographia} and Swammerdam~\cite{swammerdam1758book}, revealed that the building blocks of life are invisible to the naked human eye. 

Lens technology continuously improved through the 17\textsuperscript{th} and 18\textsuperscript{th} Centuries, revealing smaller and smaller objects, until in 1873 Abbe showed that the diffraction of light through a lens places a fundamental limit on the size of objects which can be resolved~\cite{abbe1873beitrage}. The famous Abbe diffraction limit, shown in Equation~\ref{eq:abbe}, states that the minimum separation distance, $d$, at which two objects can be resolved decreases with the wavelength of light, $\lambda$, but increases with the lens' acceptance angle, $\alpha$, and refractive index of the medium between the lens and the object, $n$. The wavelength range of visible light is \SIrange[range-phrase=--]{400}{700}{\nano\metre}, and the maximum acceptance angle of a lens approaches \SI{90}{\degree}. The refractive index of air is $1.0$, although special immersion oils can be used to increase $n$ to \textasciitilde\num{1.5}. Using Equation~\ref{eq:abbe}, we then calculate that the maximum resolution of a microscope lens is \textasciitilde\SI{200}{\nano\metre}. 

\begin{equation} \label{eq:abbe}
d = \frac{\lambda}{2n\sin\left (\alpha  \right )}\end{equation}

A number of technologies which are not based on optics exist to greatly surpass the resolution of optical microscopes, including transmission electron microscopy and scanning electron microscopy~\cite{reinhold1931configuration, wells2006early, reimer2013transmission}.
However due to sample preparation requiring a combination of freezing, fixation or a toxic staining~\cite{kuo2007electron}, as well as leathal minimum electron doses~\cite{de2016live}, these techniques are not compatible with live cell biology.

Furthermore, the invention of fluorescent labelling gives optical microscopy a distinct advantage over any other microscopy technique in terms of specificity. 
Sophisticated biochemistry, based on antibody chemistry, genetic expression, and other biotechnologies, can be used to label specific cellular compartments or organelles with fluorescent molecules~\cite{day2014fluorescent}. 
When these fluorescent molecules are illuminated with a certain wavelength of light, they absorb photons, exciting electrons to a higher energy state. 
The electrons lose some energy through so-called vibrational states, then as the electron falls back to the ground state photons are emitted at a red-shifted wavelength - where the wavelength shift is proportional to the energy lost through vibrational states, as per the Plank-Einstein relation $E=hf=hc/\lambda$~\cite[\textit{ch. 39}]{halliday2010principles}. 

Fluorescent labelling has a number of unique applications. 
Firstly, since the light used to excite fluorescence is blocked by filters before it reaches the observer, only fluorescence emission light is seen through the microscope~\cite{ploem1967use}. 
This creates a bright image of the compartment of interest against a black background, giving a high signal-to-noise ratio.
Furthermore the specificity of labelling means that other parts of the cell which are not of interest for a given experiments are invisible, further enhancing the observability of the labelled compartment~\cite{day2014fluorescent}.
Finally, if two or more fluorescent labels are used with non-overlapping fluorescent spectra, then imaging in multiple channels can be used for co-localisation studies, for example to confirm that a certain protein interacts with a certain organelle. 

The unique advantages of florescent labelling have motivated researchers in the last two decades to invent optical methods to image at resolutions below the diffraction limit~\cite{cornea2014fluorescence}.
Whilst it remains impossible to capture an individual image beyond the diffraction limit, digital cameras and intensive computational reconstruction has enabled techniques which sacrifice temporal resolution for enhanced spatial resolution.
This is the principle behind techniques based on Photoactivated Localisation Microscopy (PALM)~\cite{betzig2006imaging} and Stochastic Optical Reconstruction Microscopy (STORM)~\cite{rust2006sub}: in each image, only a small fraction (<\SI{1}{\percent}) of fluorophores are emitting light.
Assuming that two adjacent fluorophores do not emit simultaneously, the centre of the diffraction pattern is calculated as the true location of the molecule; when this is applied to a dataset of \num{1000}+ images, a super-resolution image is constructed. 

STORM is able to achieve a spatial resolution of \textasciitilde\SIrange[range-phrase=--]{10}{20}{\nano\metre}; however it takes around \SIrange[range-phrase=--]{1}{5}{\minute} to generate such an image.
Dynamic events in live cell biology cannot be captured, and so a faster technique must be used for such experiments. 

Structured illumination microscopy (SIM), as refined for super-resolution by Gustafsson, utilises 9 raw images to reconstruct a resolution-enhanced image up to twice the diffraction limit of the lens~\cite{gustafsson2000surpassing}.
Using state-of-the-art lenses and cameras, this can produce images with a spatial resolution of <\SI{100}{\nano\metre}, at a video rate of \SI{11}{\hertz}~\cite{young2016guide}.  
This microscopy technique is described in detail in Chapter~\ref{chap:LAGSIM}. 

Furthermore many modern microscopy techniques, including SIM, confocal~\cite{white1987evaluation, marvin1961microscopy}, optical projection tomography (OPT)~\cite{sharpe2002optical}, and single plane illumination microscopy (SPIM)~\cite{huisken2004optical}, are able to capture information inside the biological sample, creating a 3-dimensional (3D) volumetric dataset. 
A new tool for visualising the data produced by such microscopes is presented in Chapter~\ref{chap:FPB}. 


\section{Cell biology for microscopy}
The tools described in Chapters~\ref{chap:LAGSIM} and \ref{chap:FPB} of this document have conspicuous applications in cell biology, which are detailed in Chapters~\ref{chap:MOF} and \ref{chap:ER}.
Since this thesis covers a broad range of themes, from engineering through computer science to biology, this section is included to provide a basic understanding of cell biology necessary for appreciating the findings detailed in the application chapters. 

\subsection{Labelling organelles to visualise cell location}
Figure~\ref{fig:animal-cell} shows a generic animal cell~\cite{wikicell}. 
Highlighted in colour are 8 organelles which are utilised in this thesis. 

\begin{figure}[htbp!]
\centering
\includegraphics[width=1.0\textwidth]{animal-cell}
\captionsetup{singlelinecheck=off}
\caption[Introduction: Animal cell labelled with organelles relevant to this thesis]{A generic animal cell, adapted from Wikipedia~\cite{wikicell}, but with the organelles discussed in this thesis highlighted in colour. Numbered annotations are as follows:\newline
\begin{tabular}{p{0.3\textwidth}p{0.3\textwidth}p{0.3\textwidth}}
\begin{enumerate} 
	\item Cytosol 
	\item Cell membrane 
	\item Nucleus 
	\item Lysosome 
\end{enumerate} &
\begin{enumerate} \setcounter{enumi}{4}
	\item Endosome 
	\item Mitochondria 
	\item ER network 
	\item ER sheets 
\end{enumerate} &
\begin{enumerate} \setcounter{enumi}{8}
	\item Vesicle 
	\item Golgi apparatus 
	\item Centrosome 
	\item Nucleolus
\end{enumerate} \\
\end{tabular}} % end of caption!
\label{fig:animal-cell}
\end{figure}

Each of the organelles highlighted in Figure~\ref{fig:animal-cell} can be labelled with a fluorescent marker.
Fluorescent labelling can provide a variety of information about the cell, depending on the function of the organelles of interest~\cite{day2014fluorescent}. 

The cell nucleus contains DNA, the genetic encoding for life~\cite{alberts2002molecular}, and can be visualised under a microscope by labelling with DAPI or SiR-DNA~\cite{kapuscinski1995dapi, lukinavivcius2015sir}. 
Labelling the nucleus provides an easy method of automated cell counting, or an easy method for locating cells if it is not obvious from other labels~\cite{porter1980use}. 

The contents of the cell are contained by the cell membrane, which acts as a barrier between the contents of the cell and extra-cellular space~\cite{alberts2013essential}. 
Membrane proteins can be specifically labelled to visualise the outline of cells, to confirm whether another fluorescent object is located inside or outside the cell~\cite{yano2009tag, lee2011fluorescent, chamma2017optimized}. 

\subsection{Labelling membrane structures to observe endocytosis}
The cell can import useful raw materials into the cell through a process called endocytosis~\cite{alberts2002molecular}. 
To enter the cell, an intracellular compartment is created from the plasma membrane to contain the foreign material.
Once inside the cell, there are many pathways for processing the endocytosed contents~\cite{marsh2001endocytosis, marsh1999structural, mcmahon2011molecular}. 
A typical pathway delivers contents into small vesicles called early endosomes, which are selectively sorted to either recycle the contents to the cell membrane or develop into late endosomes. 
Late endosomes have an acidic environment, about pH\,\num{5.5}, and begin hydrolytic digestion of their contents~\cite{geisow1984ph}. 
Late endosomes mature into lysosomes with a more acidic environment and enzymes to further break down their contents~\cite{alberts2002molecular}. 
Labelling endosomes or lysosomes alongside other objects can be used for colocalisation studies, as performed in Chapter~\ref{chap:MOF}, which show whether another fluorescent substance is contained within the organelle or is free to move in the cytosol~\cite{pike2017quantifying}. 

\begin{figure}[htbp!]
\centering
\includegraphics[width=1.0\textwidth]{endocytosis-pathway}
\captionsetup{singlelinecheck=off}
\caption[Introduction: Cells import external contents through endocytosis]{A typical endocytosis pathway, adapted from ~\cite{alberts2002molecular} shows the formation of various organelles, each of which contains a different environment for ingestion and processing of the external contents. Other endocytosis pathways also occur, for example to pass the contents to ER or release it directly into the cytosol. }
\label{fig:endocytosis-pathway}
\end{figure}

In addition to endosomes and lysosomes, other internal membrane structures include the endoplasmic reticulum (ER) and the Golgi apparatus. 
Protein folding and synthesis occurs in the ER; conversely, the Golgi apparatus is involved in further protein processing and secretion of substances from the cell~\cite{dyson1978cell}. 
Content is transferred between all intracellular compartments, either through vesicles or by direct organelle-organelle interaction. 
To visualise the ER, fluorescent labels can be applied to either the ER membrane or to luminal proteins contained within the ER network~\cite{costantini2013probing}; both labelling techniques are utilised in Chapter~\ref{chap:MOF}.  

\subsection{Labelling mitochondria to investigate energy-dependent processes and cancer}
Mitochondria generate energy for the cell by converting food to adenosine triphosphate (ATP), an energy-rich molecule that is used as the basic chemical fuel to power cellular activity~\cite{alberts2013essential}. 
This process, known as the Krebs cycle, consumes oxygen and produces carbon dioxide and water as waste products, so is a type of cellular respiration. 
Furthermore, mitochondria are responsible for triggering apoptosis - that is, controlled cell death~\cite{murray1993cell}. 
In cancer cells the Krebs cycle is bypassed, and ATP is produced by a more direct process known as glycolysis~\cite{warburg1930uber}. 
By avoiding the Krebs cycle, cancer cells stop the mitochondria from triggering apoptosis, such that the cells become immortal and reproduce to form tumours~\cite{murray1993cell}. 
Chapter~\ref{chap:MOF} uses fluorescently labelled  mitochondria to assess their interaction with a drug designed to restart the Krebs cycle in cancerous cells. 

All the organelles discussed in this sectiona are located in the cytosol, a crowded solution of many different types of molecules filling the volume of the cell~\cite{goodsell1991inside}. 
The solution is so dense that the cytosol behaves as a water-based gel rather than a liquid~\cite{alberts2013essential}. 

% Finally of relevance to this thesis is the cell's cytoskeleton. 

\subsection{Labelling the cytoskeleton reveals cellular structure}
% Do I want to mention the cytoskeleton?? 

\section{Structure of this document}
Since October 2015, I have been working in the Laser Analytics Group building tools and applying them to answer biological questions. 

Chapter~\ref{chap:LAGSIM} describes the development of LAG SIM, a versatile, user-friendly structured illumination microscope. 
As well as the methods of hardware and software development, reconstruction theory is also discussed and a showcase of images is provided as a reference for capturing high-quality images. 

Chapter~\ref{chap:FPB} presents FPBioimage, an online tool for sharing and publishing 3D volumetric data. 
First developed for sharing reconstructed SIM data with a collaborator in Chicago, the chapter describes the development of the software, and presents a wide assortment of data from different 3D imaging technologies highlighting the unique features of the tool. 

Chapter~\ref{chap:MOF} demonstrates the capability of LAG SIM for facilitating hypothesis-driven research, introducing the use of metal organic frameworks as drug delivery vehicles for treating cancer. 
Three biological experiments are described showing various therapeutic effects offered by this newly developed technology. 

Chapter~\ref{chap:ER} utilises the fast, high-resolution imaging provided by LAG SIM to present pioneering discoveries about the endoplasmic reticulum. 
Computational analysis is used to understand and interpret experimental results, and compelling evidence is shown which advances the understanding of fundamental cell biology. 

Each chapter begins with its own detailed introduction to the topic, including background information and a review of relevant literature. 

Finally, a concluding Chapter~\ref{chap:conclusion} highlights the key discoveries and contributions made as part of this work.

\subsection{Aims of this document}
As a body of work, this document serves at least three purposes. 

Foremost, it is a document of evidence that I have designed and conducted my own research experiments, and have reached the level required for Doctor of Philosophy. 

Secondly, it is provided as a resource to researchers continuing my work in the Laser Analytics Group. 
Development and use of the tools is described in detail, which should allow for their continued development as new technology becomes available. 

Finally, this document provides a memento to myself of my PhD experience. 
I am proud of what I have achieved alongside my collaborators, and have included some of my favourite quotes at the start of each chapter; these simply serve as a reminder to myself that others appreciate my hard work and that the effort has undoubtedly been worth it. 


% Nomenclature command could be useful. 