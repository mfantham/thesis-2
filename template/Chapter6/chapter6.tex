%!TEX root = ../thesis.tex
%*******************************************************************************
%****************************** Third Chapter **********************************
%*******************************************************************************
\chapter{Conclusion}
\label{chap:conclusion}

% PEEL/ING is particularly important in this chapter!! 

% **************************** Define Graphics Path **************************
\ifpdf
    \graphicspath{{Chapter6/Figs/Raster/}{Chapter6/Figs/PDF/}{Chapter6/Figs/}}
\else
    \graphicspath{{Chapter6/Figs/Vector/}{Chapter6/Figs/}}
\fi

\begin{tabular}{l l}
\textbf{Joerg Bewersdorf} & So Marcus, are you an Engineer or a Scientist? \\
%\textbf{Marcus} & \ldots Can I not be both? 
\end{tabular}

\textit{Excerpt from Skype conversation, 25-01-2018}

~\newline
~\newline
~\newline

This document has detailed the development of two tools, LAG SIM and FPBioimage, and demonstrated their application to answering biological questions about utilising MOFs for drug delivery and discovering a new mechanism for luminal ER transport. 

Chapter 2 described LAG SIM, a versatile and user-friendly structured illumination microscope, capable of imaging in multiple colours at \SI{11}{\hertz} with a resolution of \SI{85}{\nano\metre}. 
Data can be captured and reconstructed with minimal training, allowing the microscope to be used for a wide variety of biological investigations. 
The SIM showcase presents several short case-study experiments, and can be used as a resource for new users deciding on the optimal set-up to answer their biological question. 

Chapter 3 presented my new volumetric rendering program, FPBioimage, with the unique selling point that it runs in a web browser.
In combination with a suite of additional software built to complement the main tool, FPBioimage makes sharing and publishing 3D volumetric data a one-click process.  
With a focus on creating a tool that is intuitive and easy to use, all researchers around the world can now immediately view their colleagues experimental results, even when separated by thousands of miles. 

Chapter 4 showed an application of these two tools for novel cancer treatments, demonstrating their capacity for facilitating hypothesis-driven research. 
The large pore size of metal organic frameworks allows them to be loaded with drugs which would otherwise be degraded in extracellular space, for example siRNA or DCA. 
We showed that these MOF complexes are successfully endocytosed by the cell, and can therefore be used as effective drug delivery systems, with the potential of treatmenting a wide range of cancers. 

Chapter 5's utilisation of fast, high resolution TIRF SIM revealed a new phenomenon in the endoplasmic reticulum, presenting evidence for tubule pinching as a method of generating directed, non-diffusive flow. 
A computational model revealed that pinching reduces the time taken for molecules to travel through the network up to 3$\times$ compared to Brownian particle motion. 
This contribution to the community's fundamental understanding of cell biology will further assist the effort to find cures for the wide range of diseases associated with ER malfunction.

My past three years work, presented in this document as a resource to those who require the tools to answer their own biological questions, has already had a meaningful impact on the community's knowledge and understanding.
This is evidenced by publications in several well-respected journals~\cite{teplensky2017temperature, fantham2017new, holcman2018single, lautenschlager2018c, moghadam2018computer}. 
Ongoing experiments resulting from these investigations give me confidence that this work will continue to provide exciting discoveries, wherever I go next. 

%\begin{flushright}
%Marcus Fantham \hspace{-6pt}
%
%\textit{Engineer}
%\end{flushright}