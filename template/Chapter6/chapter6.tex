%!TEX root = ../thesis.tex
%*******************************************************************************
%****************************** Third Chapter **********************************
%*******************************************************************************
\chapter{Conclusion}
\label{chap:conclusion}

% PEEL/ING is particularly important in this chapter!! 

% **************************** Define Graphics Path **************************
\ifpdf
    \graphicspath{{Chapter6/Figs/Raster/}{Chapter6/Figs/PDF/}{Chapter6/Figs/}}
\else
    \graphicspath{{Chapter6/Figs/Vector/}{Chapter6/Figs/}}
\fi

This thesis has detailed the development of two tools, LAG SIM and FPBioimage, and demonstrated their application to answering biological questions about utilising MOFs for drug delivery and discovering a new mechanism for luminal ER transport. 

Over the past 350\,years, optical microscopy has provided us with knowledge of the biology of life through its unique capability of observing living systems that are invisible to the naked eye. 
The last two decades have seen a number of techniques developed which surpass the diffraction limit of light, allowing new observations to further develop our understanding. 
This vital contribution was recognised with a Nobel prize in 2014 awarded to PALM/STORM and STED, two techniques which can resolve features as small as \SI{20}{\nano\metre}. 

Of course, circumventing this physical limit is not without compromise, and for these techniques temporal resolution is sacrificed for the gain in spatial resolution. 
Whilst a diffraction-limited widefield microscope can capture videos at 100 frames per second, a STORM image takes \SI{5}{\minute} for each frame to achieve its 10$\times$ resolution enhancement. 
This is not useful for observing fast dynamic events taking place within living cells. 

Structured illumination microscopy (SIM) produces images with twice the resolution of widefield images, requiring 9 raw images per resolution-enhanced frame. 
This compromise between ultimate resolution and ultimate speed gives SIM a unique capability to capture dynamic sub-cellular events. 
Furthermore, SIM can provide optical sectioning, necessary for acquisitions to be reconstructed in 3D. 

Chapter 2 described LAG SIM, a versatile and user-friendly structured illumination microscope, capable of imaging at \SI{11}{\hertz} with a resolution of \SI{80}{\nano\metre}. 
The microscope provides optical sectioning, either physically removing out-of-focus light with TIRF illumination, or computationally by attenuation of the OTF as part of the SIM reconstruction process. 
Use of an Optosplit before the camera enables simultaneous 3-colour imaging, allowing dynamic interactions of organelles to be studied without any sacrifice of either spatial or temporal resolution.

Super-resolution microscopes can be difficult to use.
There is often a steep learning curve for new users simply to acquire images. 
Even once the raw data has been captured, reconstructing it into super-resolved images usually requires expert skill and practice. 

LAG SIM has been designed to be operated by non-expert users. 
The LabVIEW interface makes capturing time-lapses, z-stacks, bookmarked cells, and large field-of-view mosaics a simple process, with the complicated synchronisation of hardware completely hidden from the user. 
Furthermore, the SIM reconstruction process has been simplified with an ImageJ plugin specifically designed for the LAG SIM, including a tool for finding optimal parameters for artefact-free reconstruction, and another for completing batch reconstruction jobs. 

LAG SIM's unique advantages over widefield microscopes, combined with its ease-of-use, have lead to a wide variety of biological investigations performed with the instrument. 
Chapter 2 included a showcase of experiments selected to show the range of applications for which LAG SIM is suited. 
Suggested parameters for the LAG SIM ImageJ plugin have been provided alongside each reconstruction as a resource to future users seeking to answer their own biological questions. 

The optical sectioning capability of LAG SIM allows extraordinary 3D datasets to be constructed, revealing information that cannot be observed in 2D slices. 
Sharing this data with collaborators around the world, however, is not a trivial tasks.
Large file sizes and complicated or expensive software make sharing 3D data a notorious problem in the microscopy community. 

Chapter 3 presented my new volumetric rendering program, FPBioimage, with the unique selling point that it runs in a web browser.
This allows remote users to explore the data interactively, viewing it from any perspective and utilising a number of advanced rendering features such as transparency and cutting planes. 
Personal exploration provides an immersive experience, giving users a much clearer understanding of the data than a 2D projection or pre-recorded video can provide. 

FPBioimage was designed with user-friendly operation as the first priority, not only for those viewing the data but also for researchers who need to share their own volumetric data. 
In combination with a suite of plugins and mobile apps built to complement the main tool, FPBioimage has made sharing and publishing 3D volumetric data a one-click process.  

FPBioimage is not restricted to the field of microscopy, and enables 
web-based rendering of any volumetric data. 
The tool has found use in a diverse range of applications, including medical imaging and computer-generated 3D simulations. 
With a focus on creating a tool that is intuitive and easy to use, all researchers around the world can now immediately view their colleagues experimental results, even when separated by thousands of miles. 

Chapter 4 detailed an application of these two tools, LAG SIM and FPBioimage, for novel cancer treatments, demonstrating their capacity for facilitating hypothesis-driven research. 

A problem of some potentially therapeutic drugs, such as siRNA or DCA, is their breakdown in extra-cellular media. 
This either means that drugs are not viable for use as a medicine, or that treatments require high doses of the drug which leads to unwelcome side effects. 
To allow these drugs to arrive inside the cell without degradation, they can be delivered inside carrier molecules. 
The large pore size of metal organic frameworks (MOFs) makes them an ideal candidate for such a drug delivery system. 

Using optical sectioning SIM, Chapter 4 showed that these MOF complexes are successfully endocytosed by the cell, and can therefore be used as effective drug delivery systems, with the potential of treating a wide range of cancers. 
All volumetric imaging data collected as part of the study was published online with FPBioimage, as an example of how the tool can be used to allow others to examine the data and confirm the findings themselves. 

One of LAG SIM's unique capabilities is fast, multi-colour TIRF microscopy in high resolution. 
This was utilised in Chapter 5 to reveal a new phenomenon in the endoplasmic reticulum (ER) for transporting proteins around the cell. 

Collaborators had observed directed laminar flow in the ER.
Acquisitions made on LAG SIM at \SI{11}{\hertz} with \SI{80}{\nano\metre} resolution revealed pinching of the ER membrane which correlates with luminal flow speed. 

This new discovery was investigated further using a computational model, which showed that pinching reduces the time taken for molecules to travel through the network up to 3$\times$ compared to Brownian particle motion. 
This contribution to the community's fundamental understanding of cell biology will further assist the effort to find cures for the wide range of diseases associated with ER malfunction.

My past three years work, presented in this document as a resource to those who require the tools to answer their own biological questions, has already had a meaningful impact on the community's knowledge and understanding.
This is evidenced by publications in several well-respected journals~\cite{teplensky2017temperature, fantham2017new, moghadam2018computer, holcman2018single, lautenschlager2018c}. 
Ongoing experiments resulting from these investigations give me confidence that this work will continue to provide exciting discoveries. 

% Put similar stuff in section 1.4 of the introduction? 