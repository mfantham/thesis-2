% ************************** Thesis Abstract *****************************
% Use `abstract' as an option in the document class to print only the titlepage and the abstract.

\begin{abstract}
Modern microscopy techniques can image beyond the diffraction limit, in three spatial dimensions, and capture sub-cellular resolution videos, providing new biological insight and assisting in drug development. 
However, such advanced instruments typically require expert engineers and physicists to operate them, limiting their throughput and practicality for answering biological questions. 
Moreover, analysis of the raw data is complicated and there are significant barriers to publishing and sharing the data with others. 

This thesis addresses these problems, presenting two tools designed to reduce the level of expertise required to acquire and publish modern microscopy data.

The development of a structured illumination microscope (SIM) is described, with a particular emphasis on control and reconstruction software designed to make SIM accessible to biologists who are new to super-resolution microscopy. 
The microscope's ease-of-use has led to a wide variety of biological investigations, which are presented as case studies to assist readers of this thesis in designing their own SIM experiments. 

The current practice for publishing 3D data is to show 2D intensity projections or fly-through videos, which present the data only from the author's perspective and do not give readers the opportunity to explore the results themselves. 
To solve this problem, Chapter~\ref{chap:FPB} introduces my new volumetric rendering program, FPBioimage, which runs in a web browser.
By creating a tool that is intuitive and easy to use, FPBioimage enables researchers around the world to immediately view their colleagues' experimental results, even when separated by thousands of miles.

Two biological studies are discussed in detail to highlight the ability of these tools to answer the latest questions in cell biology. 
SIM's combination of high speed and high resolution video capture reveals a pinching phenomenon in the endoplasmic reticulum which was previously unknown, responsible for active flow of luminal proteins.
FPBioimage is used to show metal organic frameworks successfully delivering sensitive drugs to cells, establishing a new method of cancer treatment. 

All software presented in this thesis is freely available, and has been carefully written to be reusable by other researchers. 
This is evidenced by OMERO, an online microscopy data repository, adopting FPBioimage as their default volumetric renderer. 
The open-source license under which the software is distributed means that developers can continue to build on the programs, extending the capabilities as new technology becomes available. 

\end{abstract}
