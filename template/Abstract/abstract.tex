% ************************** Thesis Abstract *****************************
% Use `abstract' as an option in the document class to print only the titlepage and the abstract.
\begin{abstract}
This work details the development of two tools demonstrates their application to answering biological questions.

Chapter 2 describes LAG SIM, a versatile and user-friendly structured illumination microscope, capable of imaging in multiple colours at \SI{11}{\hertz} with a resolution of \SI{85}{\nano\metre}. 
Data can be captured and reconstructed with minimal training, allowing the microscope to be used for a wide variety of biological investigations. 
A showcase of case-study experiments is presented, which can be used as a resource for new users to find the optimal set-up to answer their biological question. 

Chapter 3 presents my new volumetric rendering program, FPBioimage, which runs in a web browser.
In combination with a suite of additional software built to complement the main tool, FPBioimage makes sharing and publishing 3D volumetric data a one-click process.  
With a focus on creating a tool that is intuitive and easy to use, researchers around the world can now immediately view their colleagues experimental results, even when separated by thousands of miles. 

Chapter 4 shows an application of these two tools for novel cancer treatments based on metal organic frameworks (MOFs). 
The large pore size of these crystalline materials allows them to be loaded with drugs which would otherwise be degraded in extracellular space. 
Experiments show that MOF complexes are successfully endocytosed by the cell, and can therefore be used as effective drug delivery systems, with the potential of treating a wide range of cancers. 

Chapter 5 utilises fast, high resolution TIRF SIM to reveal a new phenomenon in the endoplasmic reticulum (ER). 
Evidence is presented for tubule pinching, which generates non-diffusive directed flow. 
Computational modelling reveals that pinching reduces the time taken for molecules to travel through the network up to 3$\times$ compared to Brownian particle motion. 
This contribution to the community's fundamental understanding of cell biology will further assist the effort to find cures for the wide range of diseases associated with ER malfunction.
\end{abstract}
